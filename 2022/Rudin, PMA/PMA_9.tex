\documentclass{article}
\usepackage{amsmath,amssymb,amsthm,enumitem}

\renewcommand\dim{\ensuremath{\text{dim }}}
\newcommand\bs{\ensuremath{\boldsymbol}}
\newcommand\be{\ensuremath{\boldsymbol e}}
\newcommand\bx{\ensuremath{\boldsymbol x}}
\newcommand\by{\ensuremath{\boldsymbol y}}
\newcommand\bzero{\ensuremath{\boldsymbol 0}}

\title{PMA 9 : Functions of Several Variables}
\author{}
\date{\today}

\begin{document}
\maketitle
\tableofcontents

\newpage

\setcounter{section}8

%%
\section{Linear Transformations}

%
\subsection{Definition}
\begin{enumerate}[label=(\alph*)]
\item[(d)]
\(\dim X=r\) if \(V\) contains an independent set of \(r\) vectors but contains no independent set of \(r+1\) vectors.
\item[(e)]
An independent subset of a vector space \(X\) which spans \(X\) is called a \emph{basis} of \(X\).
\end{enumerate}

%
\subsection{Theorem}
Let \(r\) be a positive integer.
If a vector space \(X\) is spanned by  a set of \(r\) vectors, then \(\dim X\le r\).

\begin{proof}
If this is false, there is a vector space \(X\) which cotains an independent set \(Q=\{\by_1,\cdots,\by_{r+1}\}\) and which is spanned by a set \(S_0=\{\bx_1,\cdots,\bx_r\}\).

\([i=0]\) :
Since \(S_0\) spans \(X\), \(\by_1\) is in the span of \(S_0\).
Hence,
\[\by_1+b_1\bx_1+\cdots+b_r\bx_r=\bzero\]
Since \(\by_1\neq\bzero\), there exists \(k_0\) such that \(b_{k_0}\neq0\)
and \(\bx_{k_0}\) can be expressed as a linear combination of \(\{\bx_1,\cdots,\bx_r,\by_1\}\setminus\{\bx_{k_0}\}\).
It follows that \(X\) is spanned by  \(\{\bx_1,\cdots,\bx_r,\by_1\}\setminus\{\bx_{k_0}\}\), which now we call \(S_1\) and rename the elements by writting \(S_1=\{\by_1,\bx_2,\cdots,\bx_r\}\).

\([i=1]\) : Since \(S_1\) spans \(X\), \(\by_2\) is in the span of \(S_1\).
Hence,
\[a_1\by_1+\by_2+b_2\bx_2+\cdots+b_r\bx_r=\bzero\]
Since \(\{\by_1,\by_2\}\) is independent, there exists \(k_1\) such that \(\bx_{k_1}\neq0\) and \(\bx_{k_1}\) can be expressed as a linear combination of \(\{\bx_2,\cdots,\bx_r,\by_1,\by_2\}\setminus\{\bx_{k_1}\}\).
It follows that \(X\) is spanned by \(\{\bx_2,\cdots,\bx_r,\by_1,\by_2\}\setminus\{\bx_{k_1}\}\), which now we call \(S_2\) and rename the elements by wrtting \(S_2=\{\by_1,\by_2,\bx_3,\cdots,\bx_r\}\).

\([i=r-1]\) :
After the step \([i=r-2]\), we have \(S_{r-1}=\{\by_1,\cdots,\by_{r-1},\bx_r\}\) which spans \(X\).
Since \(S_{r-1}\) spans \(X\), \(\by_r\)  is in the span of \(S_{r-1}\).
Hence,
\[a_1\by_1+\cdots+a_{r-1}\by_{r-1}+\by_r+b_r\bx_r=\bzero.\]
Since \(\{\by_1,\cdots,\by_r\}\) is independent, \(b_r\neq0\) and \(\bx_r\) can be expressed as a linear combination of  \(\{\by_1,\cdots,\by_r\}\).
It follows that \(X\) is spanned by \(\{\by_1,\cdots,\by_r\}\), which we call \(S_r\).

Since the set \(Q=\{\by_1,\cdots,\by_{r+1}\}\) was independent, \(\by_{r+1}\) is outside of the span of \(S_r\) or the set \(X\).
This contradiction proves the theorem.
\end{proof}

\subsection*{Corollary : \(\dim R^n=n\).}

\begin{proof}
\(R^n\) is spanned by the set \(E=\{\be_1,\cdots,\be_r\}\).
The previous theorem says that \(\dim R^n\le n\).
Moreover, \(E\) is independent set of \(n\) vectors, for which \(\dim R^n\ge n\).
\end{proof}

%
\subsection{Theorem}
Suppose \(X\) is a vectors space, and \(\dim X = n\).
\begin{enumerate}[label=(\alph*)]
\item
A set \(E\) of \(n\) vectors in \(X\) spans \(X\) if and only if \(E\) is independent.
\item
\(X\) has a basis, and every basis consists of \(n\) vectors.
\item
If \(\{\by_1,\cdots,\by_r\}\) is an independent set in \(X\) such that \(1\le r\le n\), then \(X\) has a basis containing \(\{\by_1,\cdots,\by_r\}\).
\end{enumerate}

\begin{proof}
(a)
Suppose that \(E=\{\bx_1,\cdots,\bx_n\}\) is independent and that \(\by\in X\).
The set \(E\cup\{\by\}\) is dependent since \(\dim X =n\).
That is, \(a_1\bx_1+\cdots+a_n\bx_n+b\by=\bzero\) for a set of coefficients \(a_i\)'s and \(b\), all of which is not zero.
Then, \(b\ne0\) owing to the independence of \(E\).
Thus, \(\by\) is in the span of \(E\).
Conversely, suppose that \(E\) is dependent.
Then, one of its members can be removed (to constitute a set \(E_0\) of \(n-1\) vectors) without changing the span of \(E\).
If \(E\) spans \(X\), then \(E_0\) spans \(X\) too.
Then \(\dim X\le n-1\) by the previous theorem, which is a contradiction.
Thus \(E\) spans \(X\).

(b)
Since \(\dim X = n\), \(X\) contains an independent set of \(n\) vectors.
By (a), this set spans \(X\) and is a basis of \(X\).

(c)
Note first that any subset \(A\) of \(X\), consting of more than \(n\) elements, is dependent.
For, if \(A\) were independent, then \(\dim X>n\) by \textbf{9.1.(d)}.

Now, let \(\{\bx_1,\cdots,\bx_n\}\) be a basis for \(X\).
The set \(\{\by_1,\cdots,\by_r,\bx_1,\cdots,\bx_n\}\) is a dependent set which spans \(X\).
We can remove an element \(\bx_{k_i}\) for \(1\le i\le n-r\) from the set, without changing the span of \(X\), to construct \(\{\by_1,\cdots,\by_r\}\setminus\{x_{k_1},\cdots,x_{k_{n-r}}\}\).


\end{proof}
\end{document}