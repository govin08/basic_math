\documentclass{book}
\usepackage{geometry, graphicx, kotex, imakeidx, titlesec, array}
\usepackage{amsmath, amsthm, amssymb, mathrsfs}

\geometry{paper=b5paper, left=20mm, right=20mm, top=20mm, bottom=20mm}

\titleformat{\chapter}{\normalfont\huge\bfseries}{\chaptertitlename\ \thechapter.}{20pt}{\huge} % 장 스타일 설정
\newtheorem{theorem}{Theorem} % theorem 환경
\newtheorem{definition}{Definition} % definition 환경

\makeindex % 색인 만들기

\linespread{1.25} % 줄간격 설정
\renewcommand\arraystretch{1.3} % 표 세로간격 설정

\begin{document}
\pagenumbering{gobble}

\title{선형대수}
\author{}
\date{\today}
\maketitle

\pagenumbering{roman}
\renewcommand*\contentsname{목차}
\tableofcontents

%%%
\chapter*{들어가며}

%%%
\chapter{벡터와 스칼라}

%%%
\chapter{2차원 평면}

%%%
\chapter{3차원 공간}

%%%
\chapter{유클리드 공간}

%%%
\chapter{행렬}

%%%
\chapter{일차연립방정식}

%%%
\chapter{일차변환}

%%%
\chapter{행렬식}

\newpage
\renewcommand\bibname{참고문헌}
\addcontentsline{toc}{chapter}{참고문헌}
\begin{thebibliography}{AA}
\bibitem {LSTM} Hochreiter, Sepp, and Jürgen Schmidhuber. ``Long short-term memory.'' Neural computation 9.8 (1997): 1735-1780.
\bibitem {pure} Hardy, Godfrey Harold. Course of pure mathematics. Courier Dover Publications, 2018.
\end{thebibliography}

\bigskip


\renewcommand{\indexname}{색인}
\addcontentsline{toc}{chapter}{색인}
\printindex
\bigskip

필요한 경우 색인(index)을 작성한다.


\end{document}

